\begin{table}
  \begin{tabular}{llrrrr}
\toprule
Mutation Category &   Prediction Method &    N &    R &  MAE &   FC \\
\midrule
 \multirow{ 3}{*}{Multiple mutations} & flex ddG & \multirow{ 3}{*}{273} & 0.62 & \textbf{1.62} & \textbf{0.78}  \\
 & no backrub control & & 0.58 & 1.73 & 0.73  \\
 & ZEMu paper & & \textbf{0.64} & 1.63 & 0.75  \\
\hline
 \multirow{ 3}{*}{Multiple mutations, all to alanine} & flex ddG & \multirow{ 3}{*}{191} & 0.47 & 1.77 & \textbf{0.84}  \\
 & no backrub control & & 0.50 & \textbf{1.69} & 0.81  \\
 & ZEMu paper & & \textbf{0.55} & 1.72 & 0.79  \\
\hline
 \multirow{ 3}{*}{Multiple mutations, none to alanine} & flex ddG & \multirow{ 3}{*}{45} & \textbf{0.63} & \textbf{1.38} & \textbf{0.60}  \\
 & no backrub control & & 0.44 & 1.66 & 0.58  \\
 & ZEMu paper & & 0.53 & 1.59 & \textbf{0.60}  \\
\hline
 \multirow{ 3}{*}{Mutation(s) to alanine} & flex ddG & \multirow{ 3}{*}{939} & \textbf{0.62} & \textbf{0.96} & \textbf{0.78}  \\
 & no backrub control & & 0.58 & 1.06 & 0.75  \\
 & ZEMu paper & & \textbf{0.62} & 1.03 & 0.73  \\
\bottomrule
\end{tabular}
  \caption[Multiple mutations results]{
    Multiple mutations results (backrub steps = 35000). N = number of mutations in the dataset or subset. R = Pearson's R. MAE = Mean Absolute Error. FC = Fraction Correct.
  } \label{tab:table-mult}
\end{table}
