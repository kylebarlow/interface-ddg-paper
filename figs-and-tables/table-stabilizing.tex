\begin{table}
  \begin{tabular}{llrrrr}
\toprule
Mutation Category &   Prediction Method &    N &    R &  MAE &   FC \\
\midrule
 \multirow{ 4}{*}{Stabilizing} & flex ddG & \multirow{ 4}{*}{32} & \textbf{0.62} & 2.65 & 0.00  \\
 & no backrub control & & 0.50 & 2.83 & \textbf{0.03}  \\
 & ddG monomer & & 0.39 & 2.66 & \textbf{0.03}  \\
 & ZEMu paper & & 0.31 & \textbf{2.64} & \textbf{0.03}  \\
\hline
 \multirow{ 4}{*}{Neutral} & flex ddG & \multirow{ 4}{*}{719} & \textbf{0.19} & \textbf{0.58} & \textbf{0.85}  \\
 & no backrub control & & 0.10 & 0.68 & 0.81  \\
 & ddG monomer & & 0.13 & 0.61 & 0.84  \\
 & ZEMu paper & & 0.16 & 0.64 & 0.81  \\
\hline
 \multirow{ 4}{*}{Positive} & flex ddG & \multirow{ 4}{*}{489} & \textbf{0.48} & 1.32 & 0.67  \\
 & no backrub control & & 0.44 & 1.35 & 0.70  \\
 & ddG monomer & & 0.47 & \textbf{1.31} & \textbf{0.71}  \\
 & ZEMu paper & & \textbf{0.48} & 1.32 & 0.70  \\
\bottomrule
\end{tabular}
  \caption[Flex ddG performance on stabilizing mutations]{
    Performance of the Rosetta flex ddG method on the subset of mutations experimentally determined to be stabilizing ($\Delta\Delta G <= -1$), neutral ($-1 < \Delta\Delta G < 1$), or destabilizing ($\Delta\Delta G >= 11$). Backrub steps = 35000. R = Pearson's R. MAE = Mean Absolute Error. FC = Fraction Correct. N = number of mutations in the dataset or subset.
  } \label{tab:table-stabilizing}
\end{table}
