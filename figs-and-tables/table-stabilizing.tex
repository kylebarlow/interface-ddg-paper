\begin{table}
  \begin{tabular}{llrrrr}
\toprule
Mutation Category &   Prediction Method &    N &    R &  MAE &   FC \\
\midrule
 \multirow{ 4}{*}{Stabilizing} & flex ddG & \multirow{ 4}{*}{32} & \textbf{0.54} & 2.12 & 0.09  \\
 & no backrub control & & 0.50 & 2.31 & \textbf{0.31}  \\
 & ddG monomer & & 0.39 & 2.18 & 0.19  \\
 & ZEMu paper & & 0.31 & \textbf{2.01} & \textbf{0.31}  \\
\hline
 \multirow{ 4}{*}{Neutral} & flex ddG & \multirow{ 4}{*}{719} & \textbf{0.20} & \textbf{0.51} & \textbf{0.87}  \\
 & no backrub control & & 0.10 & 0.72 & 0.78  \\
 & ddG monomer & & 0.13 & 0.75 & 0.80  \\
 & ZEMu paper & & 0.16 & 0.66 & 0.79  \\
\hline
 \multirow{ 4}{*}{Positive} & flex ddG & \multirow{ 4}{*}{489} & \textbf{0.48} & \textbf{1.55} & 0.63  \\
 & no backrub control & & 0.44 & 1.63 & 0.67  \\
 & ddG monomer & & 0.47 & 1.71 & \textbf{0.72}  \\
 & ZEMu paper & & \textbf{0.48} & 1.63 & 0.62  \\
\bottomrule
\end{tabular}
  \caption[Flex ddG performance on stabilizing mutations]{
    Performance of the Rosetta flex ddG method on the subset of mutations experimentally determined to be stabilizing ($\Delta\Delta G <= -1$), neutral ($-1 < \Delta\Delta G < 1$), or destabilizing ($\Delta\Delta G >= 1$). Flex ddG was run with 35000 backrub steps. N = number of cases in the dataset or subset. R = Pearson's R. MAE = Mean Absolute Error. FC = Fraction Correct. Best performance for each metric and dataset is shown in bold.
  } \label{tab:table-stabilizing}
\end{table}
