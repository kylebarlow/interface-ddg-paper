\begin{table}
  \begin{tabular}{llrrrr}
\toprule
Mutation Category &  Prediction Method &     N &    R &  MAE &   FC \\
\midrule
 \multirow{ 2}{*}{Complete dataset} & flex ddG & \multirow{ 2}{*}{1240} & \textbf{0.64} & 0.98 & \textbf{0.76}  \\
 & flex ddG (1.6 kT) & & \textbf{0.64} & \textbf{0.93} & 0.75  \\
\hline
 \multirow{ 2}{*}{Small-to-large mutation(s)} & flex ddG & \multirow{ 2}{*}{130} & 0.59 & 0.84 & 0.71  \\
 & flex ddG (1.6 kT) & & \textbf{0.64} & \textbf{0.81} & \textbf{0.72}  \\
\hline
 \multirow{ 2}{*}{Single mutation to alanine} & flex ddG & \multirow{ 2}{*}{748} & \textbf{0.51} & 0.77 & \textbf{0.76}  \\
 & flex ddG (1.6 kT) & & \textbf{0.51} & \textbf{0.72} & 0.75  \\
\hline
 \multirow{ 2}{*}{Multiple mutations} & flex ddG & \multirow{ 2}{*}{273} & \textbf{0.63} & 1.64 & \textbf{0.79}  \\
 & flex ddG (1.6 kT) & & \textbf{0.63} & \textbf{1.52} & 0.75  \\
\hline
 \multirow{ 2}{*}{Multiple mutations, none to alanine} & flex ddG & \multirow{ 2}{*}{45} & \textbf{0.64} & 1.42 & \textbf{0.60}  \\
 & flex ddG (1.6 kT) & & 0.62 & \textbf{1.38} & 0.58  \\
\bottomrule
\end{tabular}
  \caption[Comparison of backrub temperature results]{
    Flex ddG performance comparison, when backrub is run with a sampling temperature (kT) of 1.2 or 1.6. Backrub steps = 10000. R = Pearson's R. MAE = Mean Absolute Error. FC = Fraction Correct. N = number of mutations in the dataset or subset.
  } \label{tab:table-temperature}
\end{table}
