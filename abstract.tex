Computationally modeling changes in binding free energies post-mutation allows large scale prediction and perturbation of protein-protein binding interactions.
Additionally, methods that consider and sample greater conformational plasticity should be able to achieve higher prediction accuracy over methods that do not.
To test this hypothesis, we developed a method within the Rosetta macromolecular modeling suite that samples conformational diversity using ``backrub'' to generate an ensemble of models, then applying torsion minimization and side chain repacking and averaging across this ensemble to predict these interface \ddg\ values.
We tested our method on a curated benchmark set of 1240 mutants, and were able to outperform prior methods that sampled confromational space to a lesser degree.
We observed a large improvement on the subset of small side chain to large side chain mutations on which the consideration of backbone flexibility should be particularly beneficial.
Finally, we applied a generalized additive model (GAM) approach to the Rosetta energy function, and found a non-linear reweighting model capable of informing future energy function improvements.
