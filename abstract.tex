Computationally modeling changes in binding free energies upon mutation (interface \ddg) allows large-scale prediction and perturbation of protein-protein interactions.
Additionally, methods that consider and sample relevant conformational plasticity should be able to achieve higher prediction accuracy over methods that do not.
To test this hypothesis, we developed a method within the Rosetta macromolecular modeling suite (flex ddG) that samples conformational diversity using ``backrub'' to generate an ensemble of models, then applying torsion minimization, side chain repacking and averaging across this ensemble to estimate interface \ddg\ values.
We tested our method on a curated benchmark set of 1240 mutants, and found the method outperformed existing methods that sampled conformational space to a lesser degree.
We observed considerable improvements with flex ddG over existing methods on the subset of small side chain to large side chain mutations, as well as for multiple simultaneous non-alanine mutations, stabilizing mutations, and mutations in antibody-antigen interfaces.
Finally, we applied a generalized additive model (GAM) approach to the Rosetta energy function; the resulting non-linear reweighting model improved agreement with experimentally determined interface \ddg\ values, but also highlights the necessity of future energy function improvements.
