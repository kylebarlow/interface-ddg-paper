\documentclass{article}
\usepackage[letterpaper, portrait, margin=1in]{geometry}
\usepackage[capitalise]{cleveref}
\usepackage{import}
\usepackage{booktabs}
\usepackage{multirow}
\usepackage{csvsimple}
\usepackage{longtable}
\usepackage{graphicx}
\usepackage{listings}
\usepackage{minted}
\usepackage{caption}

\newcommand\ddg{$\Delta\Delta G$}

\usepackage{xr} % Probably not needed for submission
\externaldocument{flex-ddG}

% A listing of the contents of each file supplied as Supporting Information
% should be included. For instructions on what should be included in the
% Supporting Information as well as how to prepare this material for
% publications, refer to the journal's Instructions for Authors.

% The following files are available free of charge.
% \begin{itemize}
% \item Filename: brief description
% \item Filename: brief description
% \end{itemize}

% Supplementary figures:
% \begin{itemize}
% \item \cref{tab:table-mult}
% \item \cref{fig:steps-v-corr}
% \item \cref{tab:table-versions}
% \item \cref{fig:structs-v-corr-id-zemu-12-60000-rscript-validated-t14}
% \item \cref{fig:structs-v-corr-WildTypeComplex-ddg-monomer-16-003-zemu-2}
% \item \cref{fig:wildtypecomplex-scores-complete}
% \item \cref{fig:spear-corr-rmsd-error}
% \item \cref{fig:t14-mean-ensemble}
% \item \cref{tab:table-ref}
% \item \cref{tab:table-antibodies}
% \item \cref{fig:t14-fits-feats}
% \end{itemize}

\begin{document}

\renewcommand{\thefigure}{S\arabic{figure}}
\setcounter{figure}{0}
\renewcommand{\thetable}{S\arabic{table}}
\setcounter{table}{0}
% \renewcommand{\lstlistingname}{Supporting Listing}

% \subimport*{figs-and-tables/}{table-mult}
\subimport*{figs-and-tables/}{table-versions}
\subimport*{figs-and-tables/}{table-stabilizing}
% \subimport*{figs-and-tables/}{table-antibodies}

\clearpage
\begin{table}
  \begin{tabular}{llrrrr}
\toprule
Mutation Category &   Prediction Method &   N &     R &  MAE &   FC \\
\midrule
 \multirow{ 4}{*}{pdb-1A22} & flex ddG & \multirow{ 4}{*}{142} & \textbf{0.32} & \textbf{0.64} & 0.77  \\
 & no backrub control & & 0.18 & 0.77 & 0.74  \\
 & ddG monomer & & 0.12 & 0.91 & 0.73  \\
 & ZEMu paper & & 0.19 & 0.68 & \textbf{0.78}  \\
\hline
 \multirow{ 4}{*}{pdb-1A4Y} & flex ddG & \multirow{ 4}{*}{45} & 0.81 & 1.35 & 0.76  \\
 & no backrub control & & 0.79 & 1.47 & \textbf{0.78}  \\
 & ddG monomer & & 0.77 & 1.91 & 0.62  \\
 & ZEMu paper & & \textbf{0.87} & \textbf{1.12} & 0.73  \\
\hline
 \multirow{ 4}{*}{pdb-1ACB} & flex ddG & \multirow{ 4}{*}{6} & 0.30 & 2.85 & 0.83  \\
 & no backrub control & & 0.23 & 2.37 & 0.83  \\
 & ddG monomer & & 0.58 & \textbf{1.57} & \textbf{1.00}  \\
 & ZEMu paper & & \textbf{0.79} & 2.17 & 0.83  \\
\hline
 \multirow{ 4}{*}{pdb-1AHW} & flex ddG & \multirow{ 4}{*}{10} & -0.79 & 1.29 & 0.4  \\
 & no backrub control & & -0.42 & 1.42 & 0.4  \\
 & ddG monomer & & -0.34 & 1.26 & 0.5  \\
 & ZEMu paper & & \textbf{0.30} & \textbf{0.93} & \textbf{0.6}  \\
\hline
 \multirow{ 4}{*}{pdb-1AK4} & flex ddG & \multirow{ 4}{*}{15} & \textbf{0.69} & \textbf{0.57} & \textbf{0.73}  \\
 & no backrub control & & 0.35 & 1.01 & 0.47  \\
 & ddG monomer & & 0.63 & 1.35 & 0.60  \\
 & ZEMu paper & & 0.44 & 1.63 & 0.53  \\
\hline
 \multirow{ 4}{*}{pdb-1CBW} & flex ddG & \multirow{ 4}{*}{15} & -0.10 & \textbf{0.65} & \textbf{0.80}  \\
 & no backrub control & & \textbf{0.05} & 0.83 & 0.67  \\
 & ddG monomer & & -0.09 & 0.72 & 0.67  \\
 & ZEMu paper & & -0.26 & 0.71 & 0.67  \\
\hline
 \multirow{ 4}{*}{pdb-1CSE} & flex ddG & \multirow{ 4}{*}{6} & 0.48 & 1.87 & 0.67  \\
 & no backrub control & & 0.37 & 2.03 & 0.67  \\
 & ddG monomer & & 0.46 & 1.88 & 0.67  \\
 & ZEMu paper & & \textbf{0.87} & \textbf{0.81} & \textbf{1.00}  \\
\hline
 \multirow{ 4}{*}{pdb-1DAN} & flex ddG & \multirow{ 4}{*}{118} & 0.64 & \textbf{0.54} & \textbf{0.85}  \\
 & no backrub control & & \textbf{0.69} & 0.59 & \textbf{0.85}  \\
 & ddG monomer & & 0.61 & 0.71 & 0.83  \\
 & ZEMu paper & & 0.32 & 0.88 & 0.76  \\
\hline
 \multirow{ 4}{*}{pdb-1DFJ} & flex ddG & \multirow{ 4}{*}{20} & 0.71 & 1.26 & \textbf{0.65}  \\
 & no backrub control & & \textbf{0.83} & \textbf{1.04} & 0.60  \\
 & ddG monomer & & 0.69 & 1.38 & 0.55  \\
 & ZEMu paper & & 0.55 & 1.40 & 0.55  \\
\hline
 \multirow{ 4}{*}{pdb-1DQJ} & flex ddG & \multirow{ 4}{*}{34} & \textbf{0.44} & \textbf{1.70} & 0.76  \\
 & no backrub control & & 0.39 & 1.93 & 0.65  \\
 & ddG monomer & & 0.37 & 1.87 & \textbf{0.82}  \\
 & ZEMu paper & & 0.28 & 2.08 & 0.59  \\
\hline
 \multirow{ 4}{*}{pdb-1DVF} & flex ddG & \multirow{ 4}{*}{38} & 0.63 & 1.61 & 0.55  \\
 & no backrub control & & \textbf{0.65} & \textbf{1.50} & 0.66  \\
 & ddG monomer & & 0.61 & 1.54 & \textbf{0.71}  \\
 & ZEMu paper & & 0.57 & 1.54 & 0.53  \\
\hline
 \multirow{ 4}{*}{pdb-1E96} & flex ddG & \multirow{ 4}{*}{6} & \textbf{0.54} & 0.86 & 0.50  \\
 & no backrub control & & 0.51 & 0.91 & 0.50  \\
 & ddG monomer & & 0.45 & 0.96 & 0.50  \\
 & ZEMu paper & & 0.50 & \textbf{0.85} & \textbf{0.67}  \\
\hline
 \multirow{ 4}{*}{pdb-1EAW} & flex ddG & \multirow{ 4}{*}{27} & 0.01 & 0.59 & 0.89  \\
 & no backrub control & & 0.07 & 0.73 & 0.81  \\
 & ddG monomer & & \textbf{0.13} & 0.61 & 0.89  \\
 & ZEMu paper & & 0.00 & \textbf{0.49} & \textbf{0.93}  \\
\hline
 \multirow{ 4}{*}{pdb-1EMV} & flex ddG & \multirow{ 4}{*}{51} & \textbf{0.88} & \textbf{0.87} & \textbf{0.84}  \\
 & no backrub control & & 0.84 & 0.98 & \textbf{0.84}  \\
 & ddG monomer & & 0.84 & 0.96 & 0.80  \\
 & ZEMu paper & & 0.87 & 0.89 & \textbf{0.84}  \\
\hline
 \multirow{ 4}{*}{pdb-1F47} & flex ddG & \multirow{ 4}{*}{12} & 0.56 & \textbf{0.72} & 0.50  \\
 & no backrub control & & 0.58 & 0.87 & \textbf{0.58}  \\
 & ddG monomer & & \textbf{0.60} & 0.87 & \textbf{0.58}  \\
 & ZEMu paper & & 0.51 & 1.02 & 0.42  \\
\hline
 \multirow{ 4}{*}{pdb-1FC2} & flex ddG & \multirow{ 4}{*}{9} & 0.04 & 0.90 & 0.56  \\
 & no backrub control & & -0.09 & 1.01 & 0.67  \\
 & ddG monomer & & -0.39 & 1.19 & 0.44  \\
 & ZEMu paper & & \textbf{0.28} & \textbf{0.89} & \textbf{0.78}  \\
\hline
 \multirow{ 4}{*}{pdb-1FCC} & flex ddG & \multirow{ 4}{*}{8} & -0.29 & 1.72 & 0.38  \\
 & no backrub control & & -0.22 & 1.96 & \textbf{0.50}  \\
 & ddG monomer & & -0.06 & 1.50 & \textbf{0.50}  \\
 & ZEMu paper & & \textbf{0.16} & \textbf{1.35} & \textbf{0.50}  \\
\hline
 \multirow{ 4}{*}{pdb-1GC1} & flex ddG & \multirow{ 4}{*}{56} & 0.13 & \textbf{0.36} & \textbf{0.89}  \\
 & no backrub control & & -0.15 & 0.43 & 0.86  \\
 & ddG monomer & & 0.28 & 0.38 & 0.86  \\
 & ZEMu paper & & \textbf{0.36} & 0.55 & 0.84  \\
\hline
 \multirow{ 4}{*}{pdb-1HE8} & flex ddG & \multirow{ 4}{*}{10} & 0.05 & 0.67 & \textbf{0.6}  \\
 & no backrub control & & 0.62 & 0.91 & 0.5  \\
 & ddG monomer & & 0.26 & \textbf{0.66} & 0.3  \\
 & ZEMu paper & & \textbf{0.81} & 1.23 & 0.5  \\
\hline
 \multirow{ 4}{*}{pdb-1IAR} & flex ddG & \multirow{ 4}{*}{36} & 0.64 & \textbf{0.73} & 0.78  \\
 & no backrub control & & 0.35 & 1.32 & 0.67  \\
 & ddG monomer & & \textbf{0.66} & 0.98 & \textbf{0.81}  \\
 & ZEMu paper & & 0.45 & 0.86 & 0.78  \\
\hline
 \multirow{ 4}{*}{pdb-1JCK} & flex ddG & \multirow{ 4}{*}{7} & 0.51 & 1.15 & 0.57  \\
 & no backrub control & & 0.44 & 0.97 & \textbf{0.71}  \\
 & ddG monomer & & 0.75 & 1.25 & \textbf{0.71}  \\
 & ZEMu paper & & \textbf{0.85} & \textbf{0.94} & \textbf{0.71}  \\
\hline
 \multirow{ 4}{*}{pdb-1JRH} & flex ddG & \multirow{ 4}{*}{53} & \textbf{0.60} & \textbf{1.08} & 0.68  \\
 & no backrub control & & 0.50 & 1.25 & 0.60  \\
 & ddG monomer & & 0.52 & 1.29 & \textbf{0.75}  \\
 & ZEMu paper & & 0.57 & 1.15 & 0.58  \\
\hline
 \multirow{ 4}{*}{pdb-1JTG} & flex ddG & \multirow{ 4}{*}{118} & 0.44 & 1.87 & \textbf{0.89}  \\
 & no backrub control & & 0.39 & \textbf{1.77} & 0.83  \\
 & ddG monomer & & 0.40 & 2.12 & 0.86  \\
 & ZEMu paper & & \textbf{0.51} & \textbf{1.77} & 0.75  \\
\hline
 \multirow{ 4}{*}{pdb-1KTZ} & flex ddG & \multirow{ 4}{*}{27} & \textbf{0.86} & \textbf{0.62} & 0.74  \\
 & no backrub control & & 0.76 & 1.16 & \textbf{0.85}  \\
 & ddG monomer & & 0.71 & 1.13 & 0.81  \\
 & ZEMu paper & & 0.80 & 0.83 & 0.70  \\
\hline
 \multirow{ 4}{*}{pdb-1LFD} & flex ddG & \multirow{ 4}{*}{25} & \textbf{0.53} & \textbf{0.72} & 0.68  \\
 & no backrub control & & 0.13 & 1.21 & 0.60  \\
 & ddG monomer & & 0.28 & 0.97 & \textbf{0.72}  \\
 & ZEMu paper & & 0.27 & 0.84 & 0.60  \\
\hline
 \multirow{ 4}{*}{pdb-1MLC} & flex ddG & \multirow{ 4}{*}{16} & -0.25 & 1.04 & 0.69  \\
 & no backrub control & & 0.22 & 0.82 & 0.75  \\
 & ddG monomer & & 0.28 & 0.83 & 0.56  \\
 & ZEMu paper & & \textbf{0.79} & \textbf{0.42} & \textbf{0.88}  \\
\hline
 \multirow{ 4}{*}{pdb-1NMB} & flex ddG & \multirow{ 4}{*}{6} & 0.24 & 0.78 & \textbf{0.83}  \\
 & no backrub control & & 0.48 & 0.83 & 0.67  \\
 & ddG monomer & & 0.62 & \textbf{0.60} & \textbf{0.83}  \\
 & ZEMu paper & & \textbf{0.78} & 1.97 & 0.17  \\
\hline
 \multirow{ 4}{*}{pdb-1REW} & flex ddG & \multirow{ 4}{*}{24} & \textbf{0.90} & \textbf{0.68} & \textbf{0.92}  \\
 & no backrub control & & 0.78 & 1.19 & 0.75  \\
 & ddG monomer & & 0.76 & 1.20 & 0.79  \\
 & ZEMu paper & & 0.65 & 1.03 & \textbf{0.92}  \\
\hline
 \multirow{ 4}{*}{pdb-1S1Q} & flex ddG & \multirow{ 4}{*}{6} & 0.20 & \textbf{0.66} & \textbf{0.67}  \\
 & no backrub control & & 0.22 & 0.75 & \textbf{0.67}  \\
 & ddG monomer & & \textbf{0.34} & 0.71 & \textbf{0.67}  \\
 & ZEMu paper & & -0.07 & 1.09 & 0.50  \\
\hline
 \multirow{ 4}{*}{pdb-1TM1} & flex ddG & \multirow{ 4}{*}{21} & 0.28 & 1.76 & 0.38  \\
 & no backrub control & & 0.23 & 1.81 & 0.43  \\
 & ddG monomer & & 0.23 & 1.72 & 0.62  \\
 & ZEMu paper & & \textbf{0.58} & \textbf{1.37} & \textbf{0.71}  \\
\hline
 \multirow{ 4}{*}{pdb-1UUZ} & flex ddG & \multirow{ 4}{*}{5} & 0.83 & 0.54 & \textbf{0.8}  \\
 & no backrub control & & \textbf{0.92} & \textbf{0.52} & \textbf{0.8}  \\
 & ddG monomer & & 0.83 & 0.80 & \textbf{0.8}  \\
 & ZEMu paper & & 0.42 & 1.60 & 0.2  \\
\hline
 \multirow{ 4}{*}{pdb-1VFB} & flex ddG & \multirow{ 4}{*}{43} & 0.63 & \textbf{0.85} & \textbf{0.70}  \\
 & no backrub control & & 0.21 & 1.50 & 0.67  \\
 & ddG monomer & & \textbf{0.64} & 1.40 & \textbf{0.70}  \\
 & ZEMu paper & & 0.60 & 1.07 & 0.65  \\
\hline
 \multirow{ 4}{*}{pdb-1XD3} & flex ddG & \multirow{ 4}{*}{18} & 0.44 & 1.01 & 0.72  \\
 & no backrub control & & 0.28 & 1.43 & 0.67  \\
 & ddG monomer & & 0.51 & 1.15 & 0.67  \\
 & ZEMu paper & & \textbf{0.56} & \textbf{0.71} & \textbf{0.83}  \\
\hline
 \multirow{ 4}{*}{pdb-2I9B} & flex ddG & \multirow{ 4}{*}{5} & -0.87 & 0.56 & \textbf{1.0}  \\
 & no backrub control & & -0.00 & \textbf{0.49} & \textbf{1.0}  \\
 & ddG monomer & & -0.56 & 0.55 & \textbf{1.0}  \\
 & ZEMu paper & & \textbf{0.65} & 0.62 & \textbf{1.0}  \\
\hline
 \multirow{ 4}{*}{pdb-2JEL} & flex ddG & \multirow{ 4}{*}{43} & \textbf{0.69} & \textbf{0.61} & 0.81  \\
 & no backrub control & & 0.62 & 0.63 & 0.81  \\
 & ddG monomer & & 0.64 & 0.70 & \textbf{0.84}  \\
 & ZEMu paper & & 0.48 & 0.81 & 0.74  \\
\hline
 \multirow{ 4}{*}{pdb-2PCB} & flex ddG & \multirow{ 4}{*}{6} & \textbf{0.33} & \textbf{0.42} & \textbf{1.00}  \\
 & no backrub control & & 0.23 & 0.64 & 0.67  \\
 & ddG monomer & & -0.70 & 0.90 & \textbf{1.00}  \\
 & ZEMu paper & & -0.63 & 0.95 & 0.83  \\
\hline
 \multirow{ 4}{*}{pdb-2PCC} & flex ddG & \multirow{ 4}{*}{12} & \textbf{0.11} & \textbf{1.28} & \textbf{0.58}  \\
 & no backrub control & & 0.04 & 1.49 & 0.50  \\
 & ddG monomer & & -0.29 & 1.90 & 0.50  \\
 & ZEMu paper & & -0.24 & 1.53 & 0.50  \\
\hline
 \multirow{ 4}{*}{pdb-2VLJ} & flex ddG & \multirow{ 4}{*}{14} & \textbf{0.44} & \textbf{0.68} & \textbf{0.64}  \\
 & no backrub control & & 0.42 & 1.75 & 0.43  \\
 & ddG monomer & & 0.26 & 0.93 & 0.50  \\
 & ZEMu paper & & 0.15 & 0.89 & \textbf{0.64}  \\
\hline
 \multirow{ 4}{*}{pdb-2WPT} & flex ddG & \multirow{ 4}{*}{32} & 0.53 & 1.60 & 0.62  \\
 & no backrub control & & \textbf{0.63} & \textbf{1.39} & \textbf{0.75}  \\
 & ddG monomer & & 0.56 & 1.63 & 0.69  \\
 & ZEMu paper & & 0.45 & 1.63 & 0.56  \\
\hline
 \multirow{ 4}{*}{pdb-3BK3} & flex ddG & \multirow{ 4}{*}{13} & 0.69 & 0.57 & 0.77  \\
 & no backrub control & & \textbf{0.70} & 0.98 & 0.62  \\
 & ddG monomer & & 0.65 & \textbf{0.55} & \textbf{0.85}  \\
 & ZEMu paper & & 0.31 & 1.45 & 0.54  \\
\hline
 \multirow{ 4}{*}{pdb-3BN9} & flex ddG & \multirow{ 4}{*}{25} & 0.44 & 0.42 & \textbf{0.88}  \\
 & no backrub control & & \textbf{0.53} & \textbf{0.40} & \textbf{0.88}  \\
 & ddG monomer & & 0.31 & 0.66 & \textbf{0.88}  \\
 & ZEMu paper & & -0.09 & 0.66 & 0.84  \\
\hline
 \multirow{ 4}{*}{pdb-3NPS} & flex ddG & \multirow{ 4}{*}{27} & 0.24 & \textbf{0.67} & 0.74  \\
 & no backrub control & & \textbf{0.26} & 0.87 & \textbf{0.78}  \\
 & ddG monomer & & 0.15 & 0.84 & \textbf{0.78}  \\
 & ZEMu paper & & -0.21 & 0.89 & 0.67  \\
\bottomrule
\end{tabular}
  \caption[Flex ddG performance by structure]{
    Flex ddG performance by structure. Backrub steps = 32500. N = number of mutations in the dataset or subset. R = Pearson's R. MAE = Mean Absolute Error. FC = Fraction Correct.
  } \label{tab:table-by-structure}
\end{table}

\clearpage

\subimport*{figs-and-tables/}{structs-v-corr-id-zemu-12-60000-rscript-validated-t14}
\subimport*{figs-and-tables/}{structs-v-corr-WildTypeComplex-ddg-monomer-16-003-zemu-2}

\begin{figure}
  \centering
  \includegraphics[width=\textwidth,keepaspectratio]{figures/wildtypecomplex-scores-complete.pdf}
  \caption{
    Contour plot showing the effect of backrub sampling on the average wild-type complex score, for increasing numbers of averaged models. As the number of averaged structures N$_{struct}$ is increased along the x-axis, the average total score of the ensemble of modeled wild-type complexes (shown as colored contours in the body of the plot) also increases, as the wild-type complex models are first sorted according to their total scores and included in the averaged ensemble in order of increasing score.
    As the number of backrub sampling steps at which the ensemble is generated increases (along the y-axis), the total score of an ensemble of any number of structures tends to decrease, indicating that flex ddG is able to find lower-energy states (as measured in the Rosetta score function) as the simulation progresses. However, using only the lowest energy structures does not produce higher correlations with experimental \ddg\ values, as shown in \cref{fig:structs-v-corr-WildTypeComplex-zemu-12-60000-rscript-validated-t14}.
  } \label{fig:wildtypecomplex-scores-complete}
\end{figure}

\begin{figure}
  \centering
  \includegraphics[width=\textwidth,keepaspectratio]{figures/t14-mean-ensemble-error.pdf}
  \caption{
    Mean backrub ensemble RMSD vs. backrub steps.
  } \label{fig:t14-mean-ensemble}
\end{figure}

\begin{figure}
  \centering
  \includegraphics[width=\textwidth,keepaspectratio]{figures/t14-spear-corr.pdf}
  \caption[\ddg\ prediction error vs. ensemble RMSD]{
    Scatter plot showing the average Spearman correlation of ddG prediction error v. mean pairwise backrub ensemble RMSD, v. backrub steps.
    As mean backrub ensemble RMSD increases (\cref{fig:t14-mean-ensemble}), we don't see any significant change in correlation between mean ensemble RMSD and ddG error.
    This demonstrates that mean pairwise backrub ensemble RMSD is not an effective metric to measure the degree to which we have sampled ``enough''.
  } \label{fig:spear-corr-rmsd-error}
\end{figure}

\begin{figure}
  \centering
  \includegraphics[width=\textwidth,keepaspectratio]{figures/zemu-sigmoid2-tal-feats.png}
  \caption[Sigmoid fit Rosetta score function terms]{
    Rosetta interface \ddg\ score function terms fit on interface \ddg\ predictions via a sigmoid git Generalized Additive Model. (Figure courtesy Markus Heinonen)
  } \label{fig:t14-fits-feats}
\end{figure}

\begin{figure}
  \centering
  \includegraphics[width=\textwidth,keepaspectratio]{figures/zemu-sigmoid2-corrs.png}
  \caption[Interface \ddg prediction performance with sigmoid fit score function]{
    Left: standard, non-fitted predictions vs. experimental \ddg\ values. Right: Fit predictions vs. experimental data. Top: Control (no backrub) predictions. Middle: Backrub/talaris. Bottom: Backrub/REF. (Figure courtesy Markus Heinonen)
  } \label{fig:t14-fit-scatter}
\end{figure}

\subimport*{figs-and-tables/}{table-ref}

\clearpage

\csvreader[
longtable=|c|c|c|p{8cm}|r|,
table head=\caption{Filtered ZEMu dataset with experimental \ddg\ values}\label{tab:zemu-filtered}\\\hline
ID & PDB & Res. & Mutations & \ddg\\\hline
\endhead\hline\endfoot,
% \csvlinetotablerow\\\hline
% \endfirsthead\hline
% \csvlinetotablerow\\\hline
% \endhead\hline
% \endfoot,
]
{figures/table-zemu-filtered.csv}
{1=\DataSetID,2=\PDBFileID,3=\Resolution,4=\Mutations,5=\ExperimentalDDG}
{\DataSetID & \PDBFileID & \Resolution & \Mutations & \ExperimentalDDG}

\captionof{listing}{
  Flex ddg Rosetta Script implementation
  \label{lst:ddg-script}
}
\inputminted[
linenos=true,
% frame=single,
breaklines
]
{xml}
{listings/ddG-backrub.xml}

\clearpage

\end{document}