%%%%%%%%%%%%%%%%%%%%%%%%%%%%%%%%%%%%%%%%%%%%%%%%%%%%%%%%%%%%%%%%%%%%%
%% Start the main part of the manuscript here.
%%%%%%%%%%%%%%%%%%%%%%%%%%%%%%%%%%%%%%%%%%%%%%%%%%%%%%%%%%%%%%%%%%%%%
\section{Introduction}
This is a paragraph of text to fill the introduction of the
demonstration file.  The demonstration file attempts to show the
modifications of the standard \LaTeX\ macros that are implemented by
the \textsf{achemso} class.  These are mainly concerned with content,
as opposed to appearance.

\section{Results and discussion}

\subsection{Outline}

The document layout should follow the style of the journal concerned.
Where appropriate, sections and subsections should be added in the
normal way. If the class options are set correctly, warnings will be
given if these should not be present.

\subsection{References}

The class makes various changes to the way that references are
handled.  The class loads \textsf{natbib}, and also the
appropriate bibliography style.  References can be made using
the normal method; the citation should be placed before any
punctuation, as the class will move it if using a superscript
citation style
\cite{Mena2000,Abernethy2003,Friedman-Hill2003,EuropeanCommission2008}.
The use of \textsf{natbib} allows the use of the various citation
commands of that package: \citeauthor{Abernethy2003} have shown
something, in \citeyear{Cotton1999}, or as given by
Ref.~\citenum{Mena2000}.  Long lists of authors will be
automatically truncated in most article formats, but not in
supplementary information or reviews \cite{Pople2003}. If you
encounter problems with the citation macros, please check that
your copy of \textsf{natbib} is up to date. The demonstration
database file \texttt{references.bib} shows how to complete
entries correctly. Notice that ``\latin{et al.}'' is auto-formatted
using the \texttt{\textbackslash latin} command.

Multiple citations to be combined into a list can be given as
a single citation.  This uses the \textsf{mciteplus} package
\cite{Johnson1972,*Arduengo1992,*Eisenstein2005,*Arduengo1994}.
Citations other than the first of the list should be indicated
with a star. If the \textsf{mciteplus} package is not installed,
the standard bibliography tools will still work but starred
references will be ignored. Individual references can be referred
to using \texttt{\textbackslash mciteSubRef}:
``ref.~\mciteSubRef{Eisenstein2005}''.

The class also handles notes to be added to the bibliography.  These
should be given in place in the document \bibnote{This is a note.
The text will be moved the the references section.  The title of the
section will change to ``Notes and References''.}.  As with
citations, the text should be placed before punctuation.  A note is
also generated if a citation has an optional note.  This assumes that
the whole work has already been cited: odd numbering will result if
this is not the case \cite[p.~1]{Cotton1999}.

\subsection{Floats}

New float types are automatically set up by the class file.  The
means graphics are included as follows

\begin{figure}
  As well as the standard float types \texttt{table}\\
  and \texttt{figure}, the class also recognises\\
  \texttt{scheme}, \texttt{chart} and \texttt{graph}.
  \caption{An example figure}
  \label{fgr:example}
\end{figure}

Charts, figures and schemes do not necessarily have to be labelled or
captioned.  However, tables should always have a title. It is
possible to include a number and label for a graphic without any
title, using an empty argument to the \texttt{\textbackslash caption}
macro.

The use of the different floating environments is not required, but
it is intended to make document preparation easier for authors. In
general, you should place your graphics where they make logical
sense; the production process will move them if needed.

\subsection{Math(s)}

The \textsf{achemso} class does not load any particular additional
support for mathematics.  If packages such as \textsf{amsmath} are
required, they should be loaded in the preamble.  However,
the basic \LaTeX\ math(s) input should work correctly without
this.  Some inline material \( y = mx + c \) or $ 1 + 1 = 2 $
followed by some display. \[ A = \pi r^2 \]

It is possible to label equations in the usual way (Eq.~\ref{eqn:example}).
\begin{equation}
  \frac{\mathrm{d}}{\mathrm{d}x} \, r^2 = 2r \label{eqn:example}
\end{equation}
This can also be used to have equations containing graphical
content. To align the equation number with the middle of the graphic,
rather than the bottom, a minipage may be used.
\begin{equation}
  \begin{minipage}[c]{0.80\linewidth}
    \centering
    As illustrated here, the width of \\
    the minipage needs to allow some  \\
    space for the number to fit in to.
    %\includegraphics{graphic}
  \end{minipage}
  \label{eqn:graphic}
\end{equation}

\section{Experimental}

The usual experimental details should appear here.  This could
include a table, which can be referenced as Table~\ref{tbl:example}.
Notice that the caption is positioned at the top of the table.
\begin{table}
  \caption{An example table}
  \label{tbl:example}
  \begin{tabular}{ll}
    \hline
    Header one  & Header two  \\
    \hline
    Entry one   & Entry two   \\
    Entry three & Entry four  \\
    Entry five  & Entry five  \\
    Entry seven & Entry eight \\
    \hline
  \end{tabular}
\end{table}

Adding notes to tables can be complicated.  Perhaps the easiest
method is to generate these using the basic
\texttt{\textbackslash textsuperscript} and
\texttt{\textbackslash emph} macros, as illustrated (Table~\ref{tbl:notes}).
\begin{table}
  \caption{A table with notes}
  \label{tbl:notes}
  \begin{tabular}{ll}
    \hline
    Header one                            & Header two \\
    \hline
    Entry one\textsuperscript{\emph{a}}   & Entry two  \\
    Entry three\textsuperscript{\emph{b}} & Entry four \\
    \hline
  \end{tabular}

  \textsuperscript{\emph{a}} Some text;
  \textsuperscript{\emph{b}} Some more text.
\end{table}

The example file also loads the optional \textsf{mhchem} package, so
that formulas are easy to input: \texttt{\textbackslash ce\{H2SO4\}}
gives \ce{H2SO4}.  See the use in the bibliography file (when using
titles in the references section).

The use of new commands should be limited to simple things which will
not interfere with the production process.  For example,
\texttt{\textbackslash mycommand} has been defined in this example,
to give italic, mono-spaced text: some text.

\section{Extra information when writing JACS Communications}

When producing communications for \emph{J.~Am.\ Chem.\ Soc.}, the
class will automatically lay the text out in the style of the
journal. This gives a guide to the length of text that can be
accommodated in such a publication. There are some points to bear in
mind when preparing a JACS Communication in this way.  The layout
produced here is a \emph{model} for the published result, and the
outcome should be taken as a \emph{guide} to the final length. The
spacing and sizing of graphical content is an area where there is
some flexibility in the process.  You should not worry about the
space before and after graphics, which is set to give a guide to the
published size. This is very dependant on the final published layout.

You should be able to use the same source to produce a JACS
Communication and a normal article.  For example, this demonstration
file will work with both \texttt{type=article} and
\texttt{type=communication}. Sections and any abstract are
automatically ignored, although you will get warnings to this effect.
